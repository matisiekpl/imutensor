\documentclass[11pt]{article}
\usepackage[utf8]{inputenc}
\usepackage[T1]{fontenc}
\usepackage[polish]{babel}
\usepackage{times}
\usepackage[a4paper, total={6in, 8in}]{geometry}

\title{
\rule{\linewidth}{3pt}
Klasyfikator ruchów czujnika IMU na podstawie rekurencyjnej sieci neuronowej
\rule{\linewidth}{1pt}
}
\author{
  \textbf{Mateusz Woźniak}\\
  \texttt{wozniakmat@student.agh.edu.pl}
  \and
  \textbf{Maciej Pawłowski}\\
  \texttt{maciejp@student.agh.edu.pl}
}
\date{}

\begin{document}

\maketitle

\section{Abstrakt}

Przedmiotem tego artykułu jest omówienie realizacji zadania klasyfikacji ruchów z urządzenia IMU (Inertial Measurement Unit). Czujnik IMU określa przyśpieszenia postępowe i kątowe używając żyroskopu, akcelerometru i magnetometru. 
IMU jest powszechnie stosowane w lotnictwie, robotyce, wirtualnej rzeczywistości i medycynie. W lotnictwie umożliwia precyzyjne sterowanie statkami powietrznymi, a w robotyce wspomaga autonomiczne poruszanie się robotów. Jednym z zastosowań klasyfikatora ruchów może być detekcja przeciągnięcia samolotu. Z kolei w motoryzacji, IMU znajduje użycie w systemach kontroli stabilności pojazdów oraz w zaawansowanych systemach wspomagania kierowcy, które poprawiają bezpieczeństwo. My chcemy zaproponować realizację klasyfikatora 5 z góry ustalonych ruchów używając rekurencyjną sieć neuronowej. 
\\\\Do pomiarów wykorzystaliśmy smartfon wyposażony w czujnik IMU. Implementację modelu wykonaliśmy we frameworku PyTorch. Trening sieci neuronowej była wykonywany na Apple Macbook M1 16GB.

\subsection{Zbiór danych}


\end{document}